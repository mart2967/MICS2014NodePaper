% This is sigproc-sp.tex -FILE FOR V2.6SP OF ACM_PROC_ARTICLE-SP.CLS
% OCTOBER 2002
%
% It is an example file showing how to use the 'acm_proc_article-sp.cls' V2.6SP
% LaTeX2e document class file for Conference Proceedings submissions.
% ----------------------------------------------------------------------------------------------------------------
% This .tex file (and associated .cls V2.6SP) *DOES NOT* produce:
%       1) The Permission Statement
%       2) The Conference (location) Info information
%       3) The Copyright Line with ACM data
%       4) Page numbering
%
%  However, both the CopyrightYear (default to 2002) and the ACM Copyright Data
% (default to X-XXXXX-XX-X/XX/XX) can still be over-ridden by whatever the author
% inserts into the source .tex file.
% e.g.
% \CopyrightYear{2003} will cause 2003 to appear in the copyright line.
% \crdata{0-12345-67-8/90/12} will cause 0-12345-67-8/90/12 to appear in the copyright line.
%
% ---------------------------------------------------------------------------------------------------------------
% It is an example which *does* use the .bib file (from which the .bbl file
% is produced).
% REMEMBER HOWEVER: After having produced the .bbl file,
% and prior to final submission,
% you need to 'insert'  your .bbl file into your source .tex file so as to provide
% ONE 'self-contained' source file.
%
% Questions regarding SIGS should be sent to
% Adrienne Griscti ---> griscti@acm.org
%
% Questions/suggestions regarding the guidelines, .tex and .cls files, etc. to
% Gerald Murray ---> murray@acm.org 
%
% For tracking purposes - this is V2.6SP - OCTOBER 2002

\documentclass[12pt]{article}
\setlength{\oddsidemargin}{0in}
\setlength{\evensidemargin}{0in}
\setlength{\topmargin}{0in}
\setlength{\headheight}{0in}
\setlength{\headsep}{0in}
\setlength{\textwidth}{6in}
\setlength{\textheight}{9in}
\setlength{\parindent}{0in} 

\usepackage{graphicx} %For jpg figure inclusion
\usepackage{times} %For typeface
\usepackage{epsfig}
\usepackage{color} %For Comments
\usepackage[all]{xy}
\usepackage{float}
\usepackage{subfigure} 
\usepackage{hyperref}
\usepackage{url}
\usepackage{parskip}


%% Elena's favorite green (thanks, Fernando!)
\definecolor{ForestGreen}{RGB}{34,139,34}
\definecolor{MaxBlue}{RGB}{62,65,198}
% Uncomment this if you want to show work-in-progress comments
\newcommand{\comment}[1]{{\bf \tt  {#1}}}
% Uncomment this if you don't want to show comments
%\newcommand{\comment}[1]{}
\newcommand{\emcomment}[1]{\textcolor{ForestGreen}{\comment{Elena: {#1}}}}
\newcommand{\mmcomment}[1]{\textcolor{MaxBlue}{\comment{Max: {#1}}}}
\newcommand{\todo}[1]{\textcolor{blue}{\comment{To Do: {#1}}}}
\newcommand{\code}[1]{{\texttt {#1}}}



\begin{document}
\pagestyle{plain}
%
% --- Author Metadata here ---
%\conferenceinfo{WOODSTOCK}{'97 El Paso, Texas USA}
%\setpagenumber{50}
%\CopyrightYear{2002} % Allows default copyright year (2002) to be
%over-ridden - IF NEED BE. 
%\crdata{0-12345-67-8/90/01}  % Allows default copyright data
%(X-XXXXX-XX-X/XX/XX) to be over-ridden. 
% --- End of Author Metadata ---




\title{Adopting Node.js and Coffeescript in a Software Design Course}
%\subtitle{[Extended Abstract \comment{DO WE NEED THIS?}]
%\titlenote{}}
%
% You need the command \numberofauthors to handle the "boxing"
% and alignment of the authors under the title, and to add
% a section for authors number 4 through n.
%
% Up to the first three authors are aligned under the title;
% use the \alignauthor commands below to handle those names
% and affiliations. Add names, affiliations, addresses for
% additional authors as the argument to \additionalauthors;
% these will be set for you without further effort on your
% part as the last section in the body of your article BEFORE
% References or any Appendices.




\author{
Maxwell Marti \\
%Computer Science Discipline \\
University of Minnesota Morris\\
Morris, MN 56267\\
mart2967@morris.umn.edu
}




\date{}




\maketitle
\thispagestyle{empty}


\section*{\centering Abstract}
Making drastic changes to the contents of any computer science course is a formidable undertaking, especially when the class is based on cutting-edge software development techniques. The Software Design and Development course at the University of Minnesota, Morris is project-based, with one-third of the time at the outset of the semester going to labs intended to introduce new technologies, and the rest of the time working on a project for a customer, who may be a community member, faculty member, or even a company. The course goes through this change-over process every few years, most recently transitioning from focusing on web development with Groovy and Grails to a more modern take on the web with Node.js and Coffeescript. This report will focus on the technologies used, the development of classroom activities and learning objectives, and the systems built to support the project.
	Node.js has risen into prominence in the web development world for it’s unique properties, such as server-side JavaScript, asynchronous, non-blocking I/O, and built in web serving functionality. It is also commonly used in thick-client applications which use a front-end JavaScript framework to provide more functionality on the browser. The plethora of modules built to extend the functionality of Node is staggering, meaning that in any given project, the choice of modules will greatly influence the experience of the developer.
	Coffeescript is a language that is intended to overcome the shortcomings of JavaScript, while being universally compatible with JavaScript-based tools. It achieves this by compiling directly to JavaScript, and on average takes about one third less lines of code to write the same functionality. In addition, the generated JavaScript is as fast as natively written code, and quite readable as well.
	These two technologies comprise the core of the project, but there are numerous other choices to make in order to choose modules and libraries that suit the needs of the project. For each major part of the project, an overview will be given, with the rationale behind its selection explained in an educational and practical context.
	The number of new concepts that must be taught in order for the Node development paradigm to make sense is quite large. The methods and activities that are used to introduce students to concepts such as agile development, thick-client systems, test-driven development, version control with Git, and continuous integration using testing and deployment servers will be covered. I will also give an overview of the development environment and collaboration structure used by the students.
	In this paper I will introduce and analyze the tools chosen for the course project, the approach behind the initial lab exercises, the development workflow for the project, and the continuous integration tools used to facilitate the agile development process. I will conclude with an analysis of the effectiveness of this approach in the Software Design course.


\mmcomment{As submitted to MICS; needs to be changed}


 %\newpage
%this causes problems when compiling.

\setcounter{page}{1}


\section{Introduction}\label{sec:introduction}
\subsection{Course Background}\label{sec:course_background}
Before diving in to the technical details, the structure and goals of the course must be made clear. The Software Design and Development course is a project-based course, with students collaborating throughout the semester to build one piece of software. The students work in teams of at least four, working on one aspect of the project for the span of an iteration, typically two weeks. At the end of each iteration, the teams give presentations to the customer; this could either be a faculty member or a real company, this year's being the latter. As the semester progresses, the disparate codebases are assembled and improved by teams of increasing size, resulting in two or three large projects that must be consolidated into a final product.

\mmcomment{Maybe add a learning goals section?}

\subsection{Reasons for Transition}\label{sec:reasons_for_transition}
The course has historically been taught using different technologies, most recently the Grails framework for web application, and Java with Swing before that. The transition from a heavily server-reliant technology like Grails to a more lightweight technology was driven by the state of cutting-edge web development practices. More and more real-world web applications are being built with client-side JavaScript frameworks that handle more operations and ease the burden on the central server. These frameworks include Backbone.js, Angular.js, Ember.js, and many more. They typically do not make any assumptions about the server, other than that it can accept REST-style http requests. This leaves the developer free to choose their tools without fear of incompatibilities. The adoption of client-side JavaScript frameworks has been accelerated by both the increasing power of internet capable-devices, and the potential increased load on back-end servers due to their rapid proliferation.



\section{The Server}\label{sec:server}
The project setup will be described in two sections, covering the server and client-side technologies. The server used in this project is responsible for taking http requests, routing them to the proper functions, and returning a response. The server is primarily used as a database wrapper, it's application programming interface, or API, allows clients to access and modify persistent data.

\subsection{Node.js}\label{sec:node}
For the core of the server, Node.js was chosen. Node is a platform built for web application development, running on the V8 JavaScript engine found in Google Chrome. In contrast to traditional web servers, Node does not use threads to manage concurrent connections; instead, when a user connects, a JavaScript callback is triggered that performs the requested operation. Node uses non-blocking I/O to ensure that the server never hangs on a request and all users are served as soon as possible. The system is called an event loop,meaning that Node is always looping, listening for events and delegating them to their appropriate functions when triggered. This is similar to how a web browser handles events like clicks, hovers, changes, and button presses. Node has a massive ecosystem of plugins, called modules, that allow developers to easily add or integrate functionality they desire. Modules are installed with \code{npm}, the package manager that ships with Node. Some of the modules used in the course are detailed below. 

\subsubsection{Express}\label{sec:express}
Express is a node module (more on those below) that adds a more robust framework to Node for building web applications. It handles many common operations, allows advanced configuration of the server, and breaks up the logic into multiple areas.  For example, it simplifies the definition of API routes, defining static files directories, setting global variables, and much more. It is designed to not force developers into one type of application structure, and as a result, Express applications can take many forms.


\subsection{CoffeeScript}\label{sec:coffee}
While Node.js applications are written in JavaScript, a language called Coffeescript is used in the course. Coffeescript is a language with syntax inspired by Python, Ruby, and Haskell that transpiles to JavaScript. To transpile is to convert a high-level language into another high level language, whereas compiling converts to a lower-level language. The goal of Coffeescript is to make JavaScript easier to read and write. It accomplishes this by abstracting over the often unnecessarily complex conventions of JavaScript. A typical Coffeescript file is approximately one-third shorter than the equivalent JavaScript file. It also includes features that remove the often-unexpected nature of the JavaScript scope, automatically using the \code{var} keyword where it is needed.


\subsection{MongoDB}\label{sec:mongo}
Every web application needs a database, and MongoDB was a great fit. It stores data in JavaScript Object Notation, or JSON. This allows queries to return data that is immediately useful to the application, whereas a more traditional SQL based database might require a hefty function to parse and organize the data. Queries are also written in JavaScript. The teschnology is known for it's sharding and load balancing features , which allow one database to run across multiple machines; however, the scale of the project is not large enough to take advantage of these features.

\section{The Client}\label{sec:client}
The client refers to all the code written to run directly on the user's web browser. On initial page load, the server sends all the necessary JavaScript files to the client, which then handles layout, rendering of data to HTML, user interaction and event handling, and more. This client-side JavaScript handles many functions that were once the domain of the server, and by doing so allows the latter to focus on data operations.

\subsection{jQuery}\label{sec:jquery}
A mainstay in the web development world, jQuery is a javascript library that offers simplified scripting and interaction with HTML. The core of jQuery is the selector engine, which allows lookup, traversal, and manipulation of the Document Object Model (DOM) that makes up the heart of any web page. The other systems on the client use jQuery functions to enhance their own abilities.

\subsection{Backbone.js}\label{sec:backbone}
Backbone is a JavaScript framework for creating Single-Page Applications (SPAs). These are web pages that do not reload the whole page when asked to navigate to another part of the site. Instead, the necessary data is loaded from the server, and the relevant portion of the page is re-rendered using JavaScript. Backbone provides a structure and set of functionality for defining the operation of a site. It provides classes for Models, Views, and Collections, and Routers. A model is the raw data structure of a certain object. They can have fields, functions, and API information, among other things. Collections are simply groups of a certain type of Model. They typically also have an API route, so that Backbone can automatically download the proper set of Models when they are needed. Views are where the data is turned into HTML, and where many of the functions traditionally associated with Controllers in the MVC model have been defined. A view can represent a single Model, a Collection, or neither, but it always has a \code{render()} function that injects the HTML into the page, usually using jQuery. Views can also listen for events that occur within their scope; for example, clicking a button that is part of a view will trigger an event that the View code can then act on. The Router is the class that ties everything together. It is responsible for delegating changes in the URL to different Views, which then render the appropriate content. It is the Router that allows a SPA to have a URL structure without actually loading different pages.

\subsection{Bootstrap}\label{sec:bootstrap}
Bootstrap, formerly known as Twitter Bootstrap, is a responsive front-end framework. It provides CSS classes and JavaScript functions that help developers quickly build a user interface. It is highly customizable, and optimized for use on mobile devices. The Software Design course has used Bootstrap before, where it allowed students to focus on the functionality of the application and not worry as much about the look and feel.







%
% The following two commands are all you need in the
% initial runs of your .tex file to
% produce the bibliography for the citations in your paper.
%\bibliographystyle{abbrv}
%\end{thebibliography}




%\bibliography{generic_types}  
% You must have a proper ".bib" file
%  and remember to run:
% latex bibtex latex latex
% to resolve all references
%
% ACM needs 'a single self-contained file'!
%
\bibliographystyle{abbrv}
\bibliography{MICS2014NodePaper}









\end{document}

%%%%%%%%%%%%%%%%%%%%%%%%%%%%%%%%%%%%%%%%%%%%%%%%%%%%%%%%%%%%%%%%