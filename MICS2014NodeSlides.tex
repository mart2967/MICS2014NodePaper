\documentclass{beamer}
\usetheme{PaloAlto}
\usecolortheme{rose}

\usepackage{graphicx} %For jpg figure inclusion
\usepackage{times} %For typeface
\usepackage{epsfig}
\usepackage{color} %For Comments
\usepackage[all]{xy}
\usepackage{float}
\usepackage{subfigure} 
\usepackage{hyperref}
\usepackage{url}
\usepackage{parskip}

%% Elena's favorite green (thanks, Fernando!)
\definecolor{ForestGreen}{RGB}{34,139,34}
\definecolor{Teal}{RGB}{2,132,130}
% Uncomment this if you want to show work-in-progress comments
\newcommand{\comment}[1]{{\bf \tt  {#1}}}
% Uncomment this if you don't want to show comments
%\newcommand{\comment}[1]{}
\newcommand{\emcomment}[1]{\textcolor{ForestGreen}{\comment{Elena: {#1}}}}
\newcommand{\mmcomment}[1]{\textcolor{Teal}{\comment{Max: {#1}}}}
\newcommand{\todo}[1]{\textcolor{blue}{\comment{To Do: {#1}}}}
\newcommand{\code}[1]{{\texttt {#1}}}

%\setbeamertemplate{footline}[page number]{}


\begin{document}
\title{Adopting Node.js and CoffeeScript in a Software Design Course}
\date{\today}

\begin{frame}
\frametitle{Adopting Node.js and CoffeeScript in a Software Design Course}
{\centering
Midwest Instruction and Computing Symposium\par
April 25, 2014\par
Maxwell Marti\par
mart2967@morris.umn.edu\par
}
\end{frame}

%\frame{\frametitle{Table of contents}\tableofcontents[currentsection]} 

%\begin{frame}[fragile]
%\frametitle{Introduction}
%\end{frame}

%\begin{frame}[fragile]
%\frametitle{Overview}
%\end{frame}
\section{Introduction}
\subsection{Course Outline}
\begin{frame}[fragile]
\frametitle{The Software Design and Development Course}
	\begin{itemize}
  	 \item Single project
  	 \item Small groups
  	 \item Agile methodology
	 \item External customer: LightSide Labs
	 \item Current technology
	\end{itemize}
\end{frame}



\section{Server Technologies}
\subsection{Node.js}
\begin{frame}[fragile]
\frametitle{Node.js}
	\begin{itemize}
  	 \item Framework for network applications
  	 \item Uses the V8 JavaScript Engine
  	 \item Built in web server
	 \item Asynchronous I/O and JavaScript evaluation
	 \item Single-threaded event loop
	 \item Large number of additional modules
	\end{itemize}
\end{frame}

\begin{frame}[fragile]
\frametitle{Node Modules}
	\begin{itemize}
  	 \item Express
  	 \item Hogan
  	 \item Blanket
	\end{itemize}
\end{frame}

\subsection{CoffeeScript}
\begin{frame}[fragile]
\frametitle{CoffeeScript}
	\begin{itemize}
  	 \item A language that transpiles to JavaScript
  	 \item Clean syntax
  	 \item About 1/3 fewer lines of code
	 \item Scope safety and object encapsulation
	\end{itemize}
\end{frame}

\begin{frame}[fragile]
\frametitle{CoffeeScript to JavaScript Example}
\begingroup
    \fontsize{8pt}{12pt}\selectfont
	\begin{columns}
		\begin{column}{0.5\textwidth}
			\begin{verbatim}
num = 9
square = (x) -> x * x
numSquared = square num

myObj = 
    name: "Mr. Object",
    likes: "squaring numbers",
    mySqare: square
			\end{verbatim}
		\end{column}

		\begin{column}{0.5\textwidth}
			\begin{verbatim}
var myObj, num,
    numSquared, square;

num = 9;

square = function(x) {
  return x * x;
};

numSquared = square(num);

myObj = {
  name: "Mr. Object",
  likes: "squaring numbers",
  mySqare: square
};
			\end{verbatim}
		\end{column}
	\end{columns}
\endgroup
\end{frame}

\subsection{MongoDB}
\begin{frame}[fragile]
\frametitle{MongoDB}
	\begin{itemize}
  	 \item Document oriented database
         \item Records in JavaScript Object Notation (JSON)
  	 \item Queries in JavaScript
  	 \item Mongoose module provides extra utility
	\end{itemize}
\end{frame}

%\begin{frame}[fragile]
%\frametitle{Mongoose}
%	\begin{itemize}
% 	 \item Node.js wrapper for MongoDB
% 	 \item Document schemas
%  	 \item Ensures uniqueness and supports versioning
%	\end{itemize}
%\end{frame}

\section{Client Technologies}
\subsection{jQuery}
\begin{frame}[fragile]
\frametitle{jQuery}
	\begin{itemize}
  	 \item DOM interaction library
  	 \item Ubiquitous
  	 \item Powerful selectors and functions
	 \item Integrates with other front-end tools
	\end{itemize}
\end{frame}

\begin{frame}[fragile]
\frametitle{jQuery Example}
\begingroup
\fontsize{8pt}{12pt}\selectfont
\begin{verbatim}
// set html for element
$("p:first").html( "This is the <em>first</em> paragraph");

// set value for all inputs of myClass
$("input.myClass").val("default text");

// bind a function to the click event of an element
$("#myButton").on("click", function(){
    alert("Button was clicked");
});
\end{verbatim}
\endgroup
\end{frame}

\subsection{Backbone.js}
\begin{frame}[fragile]
\frametitle{Backbone.js}
	\begin{itemize}
  	 \item MV* Framework for client-side applications
	 \item HTML rendering on the browser
	 \item Models, Views, Routers
  	 \item Hierarchical views with event emission
	 \item REST API conformity with Models and Collections
	 \item Minimal opinionation
	\end{itemize}
\end{frame}

\subsection{Bootstrap}
\begin{frame}[fragile]
\frametitle{Bootstrap}
	\begin{itemize}
  	 \item Formerly Twitter Bootstrap
	 \item Responsive grid layouts
	 \item CSS classes and JavaScript utilities for the client
  	 \item Easy to customize
	 \item Simplifies web interface design
	\end{itemize}
\end{frame}

%\begin{frame}[fragile]
%\frametitle{Git and GitHub}
%	\begin{itemize}
%  	 \item Git is an advanced version control tool
%	 \item GitHub is a web service for managing Git repositories
%  	 \item Integrated in WebStorm
%	 \item Nontrivial learning curve
%	\end{itemize}
%\end{frame}

\section{Testing and Integration}

\subsection{Course Workflow}
\begin{frame}[fragile]
\frametitle{Development Workflow}
	\begin{itemize}
  	 \item WebStorm IDE (based on IntelliJ IDEA)
	 \item Unit testing with Mocha
  	 \item Version control with GitHub
  	 \item Continuous integration with Strider
	\end{itemize}
\end{frame}

\subsection{Strider}
\begin{frame}[fragile]
\frametitle{Strider}
	\begin{itemize}
  	 \item Continuous integration and deployment server
	 \item Pulls code from GitHub
  	 \item Compiles Coffeescript and runs tests
	 \item Computes code coverage metrics
	 \item Deploys projects to web server for evaluation
  	 \item Supports plugins and custom shell scripts
	\end{itemize}
\end{frame}

\subsection{Mocha}
\begin{frame}[fragile]
\frametitle{Unit Testing With Mocha}
	\begin{itemize}
  	 \item Defines structure and syntax of test code
	 \item Choice of assertion libraries
  	 \item Different testing styles supported
	 \item Versions for Node.js and client-side code
	\end{itemize}
\end{frame}

\section{Conclusion}
\begin{frame}[fragile]
\frametitle{Conclusion and Observations}
	\begin{itemize}
  	 \item Unified language both helps and hurts students
	 \item Particular difficulty with Backbone.js
  	 \item Module driven development
	 \item High velocity and low familiarity affects testing practices
  	 \item Customer interaction
	\end{itemize}
\end{frame}


\end{document}